\documentclass[iop]{emulateapj-rtx4}
%\documentclass[twocolumn,tighten]{aastex6}
%\documentclass{aastex6}
%\usepackage{emulateapj-rtx4}
%\usepackage{emulateapj}

 \shortauthors{French $\&$ Wakker}

\usepackage{graphicx}
\usepackage{subfigure}
\usepackage{hyperref}
\usepackage{amsmath}

%\usepackage{amssymb}
%\usepackage{wrapfig}
%\usepackage{setspace}

%\usepackage{mathtools}
%\frenchspacing

\newcommand{\kms}{$\rm km\, s^{-1}$}



\graphicspath{{figures//}}

\begin{document}

\title{Ly$\alpha$ absorbers do/not co-rotate with galaxy disks}

\author{David M. French, Bart P. Wakker}

\affil{Department of Astronomy, University of Wisconsin, Madison, WI 53706, USA}

\begin{abstract}

We present results of a study comparing the relative velocity of Ly$\alpha$ absorbers to the rotation direction and velocity of nearby galaxy disks. We find...

\end{abstract}


\keywords{galaxies:intergalactic medium, galaxies:evolution, galaxies:halos, quasars: absorption lines}


\section{INTRODUCTION}
Galaxy rotation curves have been observed to extend at constant velocity out to... (cite...). It becomes increasingly difficult to measure gas rotation much farther from this however, as the... Within this region the galaxy disks transition into circumgalactic medium (CGM), and eventually the CGM merges with the intergalactic medium (IGM). At what point, however, does the surrounding medium cease to circulate with the galaxy? Stewart et al. (2011) suggests through (HYDRO?) simulations that the bulk CGM kinematics out to (WHAT DISTANCE) may circulate, and that absorption in intervening QSO sightlines should be able to accurately capture this rotation signature.

There have been several studies with a sample size of 1 or a few aiming to compare the kinematics of the galaxy disk to absorption detected in it's CGM halo (e.g., Wakker \& Savage 2009, Bowen et al. 2016, \textbf{MORE}). With these individual results we may be missing the forest for the sake of the individual trees. There has yet to be a more systematic search for observational evidence that the CGM is kinematically associated with galaxies in general.


\begin{table*}[ht]\footnotesize
\begin{center}
\begin{tabular}{l l l l l l l l l l}
 \hline \hline
  Target 		& R.A. 		& Dec. 		& \textit{z}		 & Program 	  & Grating 	  & Obs ID 	    & Obs Date 	    & $T_{exp}*$     & S/N*  \\ 
  	    		& 	       		&	  		& 		  	 & 		    	  & 		  	  & 		  	   & 		     	    & 	        [ks]        & [1238] \\ 
 \scriptsize (1)  & \scriptsize (2) & \scriptsize (3) & \scriptsize (4) & \scriptsize (5) & \scriptsize (6) & \scriptsize  (7) & \scriptsize (8) & \scriptsize (9) & \scriptsize (10)  \\ \hline \hline
\\
    
1H0717+714		  &  7.0  21.0   53.3  &     71.0  20.0  36.0  &    0.5003  & 12025  	    &   G130M  &   LBG812  		 & 11-12-27      	 	  &  6.0    &      37         \\

 \\
\hline

\end{tabular}
\end{center}
  \caption{\small{COS targets in this sample. *Total exposure time and S/N ratio is given for multi-orbit exposures.}}
  \label{target_table}
\end{table*}


\section{DATA AND ANALYSIS}

\subsection{SALT Data}
Our sample contains 12 galaxies observed with the Southern African Large Telescope (SALT) Robert Stobie Spectrograph (RSS) in longslit mode. These 12 were selected from a larger pool of 48 submitted targets by the SALT observing queue. These 48 possible targets were chosen for their proximity to background QSOs whose spectra contained promising Ly$\alpha$ lines. Finally, we only included galaxies with $z \leq 0.33$ ($cz \leq 10,000$ \kms), angular sizes less than 6' to ensure easy sky subtraction, and surface brightnesses sufficient to keep exposure times below $~1300 s$. Table \ref{salt_targets} summarizes these observations. Data was taken for 2 additional galaxies, NGC3640 and NGC2962, but proved unusable due to issues with spectral identification and signal to noise (respectively).



%All SALT galaxy spectra were reduced and extracted using the standard PySALT reduction package (\textbf{\begin{table*}[ht]\footnotesize. 


All SALT galaxy spectra were reduced and extracted using the standard PySALT reduction package (\textbf{CITATION}), which includes procedures to prepare the data, correct for gain, cross-talk, bias, and overscan, and finally mosaic the images from different extensions. Next, we rectify the images with wavelength solutions found via Ne and Ar arc lamp spectra line identification. Finally, we perform a basic sky subtraction using an off-sky portion of the image, and extract 5-10 pixel wide 1-D strips from the reduced 2-D spectrum. 

For each 1-D spectrum, we identify the H$\alpha$ emission or Ca H\&K absorption lines and perform a non-linear least-squares Voigt profile fit using the Python package LMFIT\footnote{\url{http://cars9.uchicago.edu/software/python/lmfit/contents.html}}. The line centroid and 1$\sigma$ standard errors are returned, and these fits are then shifted to rest-velocity based on the galaxy systemic redshift and heliocentric velocity corrections are calculated with the IRAF rvcorrect procedure. The final rotation velocity is calculated by then applying the inclination correction, $v_{rot} = v / \sin(i)$. Final errors are calculated as

\begin{equation}
\begin{split}
	\sigma^2 = \left( \frac{\partial v_{rot}}{\partial \lambda_{obs}} \right)^2 (\Delta \lambda_{obs})^2 + \left(\frac{\partial v_{rot}}{\partial v_{sys}} \right)^2 (\Delta v_{sys})^2 + \\
	\left( \frac{\partial v_{rot}}{\partial i} \right)^2 (\Delta i)^2,
\end{split}
\end{equation}

\noindent where $\Delta \lambda_{obs}$, $\Delta v_{sys}$, and $\Delta i$ are the errors in observed line center, galaxy redshift, and inclination, respectively. The final physical scale is calculated using the SALT image scale of 0.1267 arcsec/pixel, multiplied by the 4-pixel spatial binning, and converted to physical units using a redshift-independent distance if available, and a Hubble flow estimate if not. We adopt a Hubble constant of $H_0$ = 71 \kms $\rm Mpc^{-1}$ throughout.

Finally, we calculate our approaching and receding velocities via a weighted mean of the outer 1/2 of each rotation curve, with errors calculated as weighted standard errors in the mean. Our final redshifts are calculated by forcing symmetric rotation, such that the outer 1/2 average velocity for each side matches. See Figure \ref{rotation_curve} for an example.

\begin{figure}[b!]
        \centering
        \vspace{0pt}
        \includegraphics[width=0.50\textwidth]{NGC3633_rotation_curve_xphys_helio_vemit.pdf}
        \caption{\small{Rotation curve of NGC3633. The solid green line indicates the weighted mean velocity over the corresponding x-axis region, and the shaded green indicates the 1$\sigma$ error in the mean.}}
        \label{completeness}
\end{figure}



\begin{table*}[ht]\footnotesize. 
\begin{center}
\begin{tabular}{l l l l l l l l l l l}
 \hline \hline
  Galaxy 		& R.A. 		& Dec. 		 	& \textit{cz}	& Type		& Grating		& LOS Velocity		& Absolute Velocity	& Obs Date	& $T_{exp}$		& S/N			\\ 
  	    		& 	       		&	  		 	& (\kms)		& 		  	&			& (\kms)		     	& (\kms)			&			& (ks)			& (6562.8)			\\ 
 \scriptsize (1)  	& \scriptsize (2)	& \scriptsize (3) 	&\scriptsize (4)	& \scriptsize (5)	& \scriptsize (6)	& \scriptsize (7)		& \scriptsize (8)		& \scriptsize (9)	& \scriptsize (10)	& \scriptsize (11)	\\ \hline \hline
 
 NGC4536	& 12 34 27.05	& +02 11 17.3		& $1808 \pm1$	& SAB(rc)bc	& PG2300		& $-107 \pm 8.5$	& $-113 \pm 9.2$	& 05 11 2016	& 1300			& not sure \\
  			&			&				&			&			&			& $145 \pm 31.8$	& $113 \pm 34.3$	&			&				&		\\
			
 NGC3633	& 11 20 26.22	& +03 35 08.2		& $2600 \pm2$	& SAa		& PG2300		& $-160 \pm 5.7$	& $-169 \pm 6.0$	& 05 11 2016	& 1200			& not sure \\
 			&			&				&			&			&			& $139 \pm 3.3$	& $146 \pm 3.5$	&			&				&		\\
			
 NGC5786	& 14 58 56.26	& $-$42 00 48.1	& $2998 \pm5$	& (R')SB(s)bc	& PG2300		& not sure			& not sure			& 05 11 2016	& 250			& not sure \\
  			&			&				&			&			&			& $ \pm $	& $ \pm $	&			&				&		\\

 NGC5364	& 13 56 12.00	& +05 00 52.1		& $1241 \pm4$	& SA(rs)bc pec	& PG2300		& not sure			& not sure			& 05 11 2016	& 700			& not sure \\
 	
 
 
 \hline

\end{tabular}
\end{center}
  \caption{\small{SALT targeted galaxies. Columns are as follows: 1) the galaxy name, 2), 3) R.A., Dec. in J2000, 4) galaxy systemic velocity, 5) morphological type (RC3), 6) RSS grating used, 7) approaching side velocity, 8) receding side velocity, 9) observation date, 10) exposure time, and 11) S/N of the H$\alpha$ or Ca H\&K lines.}}
  \label{salt_targets}
\end{table*}

\subsubsection{CGCG039-137}
Systemic velocity as published: 6902
Velocity as measured: 6917.8 $\pm$ 23.7
Rotation velocity (inc corrected) 139 $\pm$ 26 \kms
Rotation velocity (observed) 132 $\pm$ 16 \kms
Inclination: 61
Adjusted Inc: 63
Morphology: Scd
L_{\**} = 0.62 \\

Two sightlines: \\
RX_J1121.2+0326 at 99 kpc, 71deg az: \\
6975 Lya (dv = 75 \kms on pos side)

SDSSJ112224.10+031802.0 at 491 kpc, 24deg az : \\
6606 Unmarked (dv = -312 \kms on neg side)



\subsubsection{ESO343-G014}
Systemic velocity as published: 9162
Velocity as measured: 9138.9 $\pm$ 31.7
Rotation velocity (inc corrected) 205 $\pm$ 53 \kms
Rotation velocity (observed) 203 $\pm$ 6 \kms
Inclination: 84
Adjusted Inc: 90
Morphology: Sb
L_{\**} = 1.1 \\

One sightline: \\
RBS1768 at 466 kpc, 74deg az: \\
9308 Lya (dv = 169 \kms on pos side)
9360 Lya (dv = 221 \kms on pos side)
9434 Lya (dv = 295 \kms on pos side)


\subsubsection{IC5325}
Systemic velocity as published: 1503
Velocity as measured: 1511.9 $\pm$ 8.4
Rotation velocity (inc corrected) 125 $\pm$ 45 \kms
Rotation velocity (observed) 53 $\pm$ 5 \kms
Inclination: 25
Adjusted Inc: 25
Morphology: SAB(rs)bc
L_{\**} = 0.9 \\

One sightline: \\
RBS2000 at 314 kpc, 64deg az: \\
1598 Lya (dv = 86 \kms on possibly? pos side)


\subsubsection{MCG-03-58-009}
Systemic velocity as published: 9030
Velocity as measured: 9014.9 $\pm$ 18.6
Rotation velocity (inc corrected) 171 $\pm$ 24 \kms
Rotation velocity (observed) 150 $\pm$ 12 \kms
Inclination: 48
Adjusted Inc: 49
Morphology: Sc
L_{\**} = 2.9 \\

One sightline: \\
MRC2251-178 at 355 kpc, 71deg az: \\
9029 Lya (dv = 14 \kms on pos side)


\subsubsection{MCG-03-58-009}
Systemic velocity as published: 9030
Velocity as measured: 9014.9 $\pm$ 18.6
Rotation velocity (inc corrected) 171 $\pm$ 24 \kms
Rotation velocity (observed) 150 $\pm$ 12 \kms
Inclination: 48
Adjusted Inc: 49
Morphology: Sc
L_{\**} = 2.9 \\

One sightline: \\
MRC2251-178 at 355 kpc, 71deg az: \\
9029 Lya (dv = 14 \kms on pos side)




\subsubsection{NGC1566}
Systemic velocity as published: 1504
Velocity as measured: 1501.9 $\pm$ 14.9
Rotation velocity (inc corrected) 86 $\pm$ 21 \kms
Rotation velocity (observed) 64 $\pm$ 13 \kms
Inclination: 46
Adjusted Inc: 48
Morphology: (R'_1)SAB(rs)bcSy1
L_{\**} = 0.59
Four sightlines: 
1H0419-577 at 303 kpc, 10deg az: \\
1071 Lya (dv = -427 \kms on pos side)
1123 Lya (dv = -379 \kms on pos side)
1188 Lya (dv = -314 \kms on pos side)
1264 Lya (dv = -238 \kms on pos side)
2020 Lya (dv = 518 \kms on pos side)

HE0429-5343 at 256 kpc, 60deg az: \\
1167 Lya (dv = -335 \kms on neg side)
1358 Lya (dv = -144 \kms on neg side)

HE0435-5304 at 396 kpc, 62deg az: \\
not finished

RBS567 at 423 kpc, 69deg az: \\
1664 Lya (dv = 162 \kms on neg side)

HE0439-5254 at 459 kpc, 65deg az: \\
not finished



\subsubsection{NGC3513}
Systemic velocity as published: 1194
Velocity as measured: 1203.7 $\pm$ 12.0
Rotation velocity (inc corrected) 20 $\pm$ 22 \kms
Rotation velocity (observed) 11 $\pm$ 9 \kms
Inclination: 30
Adjusted Inc: 30
Morphology: SB(s)c_HII
L_{\**} = 0.49 \\

One sightline: \\
H1101-232 at 60 kpc, 67deg az: \\
1182 Lya (dv = -22 \kms on pos side)


\subsubsection{NGC3633}
Several locations show two velocities for emission. We have combined these into a single velocity measurement via a weighted average. We measure a redshift for this galaxy of $cz = 2597.6 \pm 2.4$ \kms.

We measure a line-of-sight rotation velocity for NGC3633 of $v_{rot}=139\pm 3.3,~-160\pm5.7$,  \kms.


Systemic velocity as published: 2600
Velocity as measured: 2587.2 $\pm$ 6.6
Rotation velocity (inc corrected) 157 $\pm$ 11 \kms
Rotation velocity (observed) 149 $\pm$ 6 \kms
Inclination: 69
Adjusted Inc: 72
Morphology: SAa:_sp_HII
L_{\**} = 0.88 \\

Three sightlines: \\
SDSSJ112005.00+041323.0 at 468 kpc, 78deg az: \\
2285 Lya (dv = -302 \kms on neg side)
2578 Lya (dv = -9 \kms on neg side)


RX_J1121.2+0326 at 184 kpc, 58deg az: \\
2605 Lya (dv = 18 \kms on neg side)


SDSSJ112224.10+031802.0 at 413 kpc, 50deg az: \\
Nothing


\subsubsection{NGC4536}
The data on the receding side of NGC4536 is very messy, and may include contamination from background sources. 

%However, the approaching side is well sampled and stable with a value of $v_{rot}=-75pm9$ \kms. For this reason, and assuming symmetry for this grand-design spiral galaxy, we have decided to adopt the same value for the receding side, $v_{rot}=75pm9$ \kms.

% The image of this galaxy is confusing - it looks opposite, because we're 'underneath' it (i.e. the near edge is up, the far edge is away). 

Systemic velocity as published: 1808
Velocity as measured: 1866.9 $\pm$ 32.9
Rotation velocity (inc corrected) 139 $\pm$ 37 \kms
Rotation velocity (observed) 129 $\pm$ 32 \kms
Inclination: 59
Adjusted Inc: 61
Morphology: SAB(rs)bc;HII_Sbrst
L_{\**} = 2.0 \\

Three sightlines: \\
3C273.0 at 349 kpc, 11deg az: \\
1580 Lya (dv = -287 \kms on pos side)
2156 Lya (dv = 289 \kms on pos side)
2267 Lya (dv = 400 \kms on pos side)


HE1228+0131 at 338 kpc, 51deg az: \\
1495 Lya (dv = -372 \kms on pos side)
1571 Lya (dv = -296 \kms on pos side)
1686 Lya (dv = -181 \kms on pos side)
1721 Lya (dv = -146 \kms on pos side)
1854 Lya (dv = -13 \kms on pos side)
2311 Lya (dv = 444 \kms on pos side)

SDSSJ123748.99+012607.0 at  294 kpc, 37deg az: \\
not finished


\subsubsection{NGC4939}
Systemic velocity as published: 3110
Velocity as measured: 3092.8 $\pm$ 33
Rotation velocity (inc corrected) 275 $\pm$ 49 \kms
Rotation velocity (observed) 204 $\pm$ 25 \kms
Inclination: 46
Adjusted Inc: 48
Morphology: SA(s)bc_Sy2
L_{\**} = 5.5 \\

One sightline: \\
PG1302-102 at 254 kpc, 61deg az: \\
3448 Lya (dv = 355 \kms on neg side)



\subsubsection{NGC5364}
Systemic velocity as published: 1241
Velocity as measured: 1238.0 $\pm$ 16.9
Rotation velocity (inc corrected) 155 $\pm$ 27 \kms
Rotation velocity (observed) 130 $\pm$ 13 \kms
Inclination: 55
Adjusted Inc: 57
Morphology: SA(rs)bc_pec_HII
L_{\**} = 1.9 \\

Two sightline: \\
SDSSJ135309.50+033328.0 at 519 kpc, 21deg az: \\
not finished

SDSSJ135726.27+043541.4 at 165 kpc, 84deg az: \\
1124 Lya (dv = -114 \kms on pos? side)
1296 Lya (dv = 58 \kms on pos? side)



\subsubsection{NGC5786}
Systemic velocity as published: 2998
Velocity as measured: 2974.6 $\pm$ 21.5
Rotation velocity (inc corrected) 172 $\pm$ 28 \kms
Rotation velocity (observed) 156 $\pm$ 19 \kms
Inclination: 63
Adjusted Inc: 65
Morphology: (R'_2)SAB(s)bc
L_{\**} = 25 \\

One sightline: \\
QSO1500-4140 at 453 kpc, 1deg az: \\
3141 Lya (dv = 166 \kms on pos side)


\subsubsection{UGC09760}
Systemic velocity as published: 2023
Velocity as measured: 2093.7 $\pm$ 15.5
Rotation velocity (inc corrected) 46 $\pm$ 16 \kms
Rotation velocity (observed) 46 $\pm$ 12 \kms
Inclination: 85
Adjusted Inc: 90
Morphology: Sd
L_{\**} = 0.17 \\

Two sightlines: \\
SDSSJ151237.15+012846.0 at 123 kpc, 90deg az: \\
2051 Lya (dv = -43 \kms on minor axis. Looks neg side, but extremely close)



\subsection{Ancillary Data}


\subsection{Galaxy Data}





%\begin{figure}[ht!]
%        \centering
%        \vspace{0pt}
%        \includegraphics[width=0.50\textwidth]{fig1.pdf}
%        \caption{\small{Distribution of $L/L_{\**}$ values for all galaxies in the dataset. Black vertical lines highlight 1, 0.5, 0.1, 0.05 and 0.01 $L_{\**}$. The turnoff around 0.1$L_{\**}$ shows that on average, the dataset is mostly complete to 0.2$L_{\**}$.}}
%%        \vspace{-5pt}
%        \label{completeness}
%\end{figure} 


\subsection{Spectra}

\begin{table*}[ht]\footnotesize
\begin{center}
\begin{tabular}{l l l l l l l l l l l l l l l}
 \hline \hline
  $Target$	&  $Galaxy$  & $R_{vir}$        & $v_{galaxy}$ 	   	  &  $Inc.$               &  $Az.$ 	       & $\rho$		   & $v_{Ly\alpha}$	 	  	& $W_{Ly\alpha}$  & $\Delta v$  			 & $\mathcal{L}$ \\ 
  	  	&       & \scriptsize (kpc) & \scriptsize  $\rm (km ~s^{-1})$ & \scriptsize (deg) & \scriptsize [deg] & \scriptsize (kpc) & \scriptsize  $\rm (km\, s^{-1})$ & \scriptsize $\rm (km\, s^{-1})$ & \scriptsize  $\rm (km\, s^{-1})$ &  \\
 \scriptsize (1) & \scriptsize (2) & \scriptsize (3)    & \scriptsize (4)     & \scriptsize (5)    & \scriptsize (6)   & \scriptsize  (7)   & \scriptsize (8) & \scriptsize (9) & \scriptsize (10) & \scriptsize (11) \\ \hline \hline

1H0717+714  &  UGC03804  &  173  &  2887  &  55  &  7  &  207  &  2870  &  343$\pm$6  &  17  &  0.24  \\


 \\
\hline
\end{tabular}
\end{center}
  \caption{\small{All associated systems. The largest $\mathcal{L}$ value is given, with a (\**) indicating that this corresponds to $\mathcal{L}_{d^{1.5}}$, otherwise the quoted $\mathcal{L}$ was computed with $R_{vir}$.}}
  \label{target_table}
\end{table*}


\section{RESULTS}


%\begin{figure}[t!]
%\centering
%  \subfigure[]{\includegraphics[width=0.87\linewidth]{fig2.pdf}}{\label{line}}
%  \subfigure[]{\includegraphics[width=1.\linewidth]{fig3.pdf}\label{impactmap}}
%  \caption{\small{a) An example of 2 Ly$\alpha$ lines found in the Mrk290 sightline at 3090 and 3192 . b) A map of \textit{all} galaxies within a 500 kpc impact parameter of target Mrk290 sightline and with velocity ($cz$) within 400 $\rm km\, s^{-1}$ of absorption detected at 3192 $\rm km\, s^{-1}$ (central black star). The galaxy NGC5987 ($v=3010$ $\rm km\, s^{-1}$, inclination = $65^{\circ}$) can be unambiguously paired with the Ly$\alpha$ absorption features at $v=3090, 3192$ $\rm km\, s^{-1}$ because it is the largest and closest galaxy in both physical and velocity space to the absorption feature.}}
%\vspace{5pt}
%\end{figure}


To facilitate this decision, we calculate the likelihood, $\mathcal{L}$, of every possible galaxy-absorber pairing as follows:

\begin{equation}
	\mathcal{L} = A e^{-(\frac{\rho}{R_{eff}})^2} e^{-(\frac{\Delta v}{200})^2}.
\end{equation}

\noindent Here $\rho$ is the physical impact parameter, $\Delta v$ the velocity difference between the absorber and the galaxy ($\Delta v = v_{galaxy} - v_{absorber}$), and $A$ is a factor included to increase the likelihood in the case that $\rho \leq R_{eff}$ (in which case $A = 2$, otherwise $A = 1$). 



\section{SUMMARY}



\begin{table}[ht]\footnotesize
\begin{center}
\begin{tabular}{l l l}
 \hline \hline
 Statistic                				&  Blueshifted Absorbers   &     Redshifted Absorbers     \\ 
  \hline \hline
 Number 	          			 		&     	22				&	26			\\
 Mean $EW$    \scriptsize $\rm [m\AA]$    &	$329 \pm 52$ 		&	$245 \pm 34$  	\\
  
\hline
\end{tabular}
\end{center}
  \caption{\small{Average properties of the associated galaxy sample split into red and blue-shifted bins based on $\Delta v$.}}
  \label{resultsTable}
\end{table}


\vspace{10pt}

\indent \textbullet \indent First result


\acknowledgements

This research has made use of the NASA/IPAC Extragalactic Database (NED) which is operated by the Jet Propulsion Laboratory, California Institute of Technology, under contract with the National Aeronautics and Space Administration. Based on observations with the NASA/ESA \textit{Hubble Space Telescope}, obtained at the Space Telescope Science Institute (STScI), which is operated by the Association of Universities for Research in Astronomy, Inc., under NASA contract NAS 5-26555. \textbf{SALT ACKNOWLEDGEMENT}. Spectra were retrieved from the Barbara A. Mikulski Archive for Space Telescopes (MAST) at STScI. Over the course of this study, D.M.F. and B.P.W. were supported by grant AST-1108913, awarded by the US National Science Foundation, and by NASA grants \textit{HST}-AR-12842.01-A, \textit{HST}-AR-13893.01-A, and \textit{HST}-GO-14240 (STScI). 

\facility{HST (COS)}


\nocite{*}
\bibliography{rotation_bib}
\bibliographystyle{apj}

\end{document}
