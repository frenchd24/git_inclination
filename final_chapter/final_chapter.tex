%\documentclass[iop]{emulateapj-rtx4}
% \shortauthors{French $\&$ Wakker}
%
%\usepackage{graphicx}
%\usepackage{subfigure}
%\usepackage{hyperref}
%\usepackage{amsmath}


%%%%%%%%%%
\documentclass[twocolumn,tighten]{aastex62}
%\documentclass{aastex6}
%\usepackage{emulateapj-rtx4}
%\usepackage{emulateapj}

 \shortauthors{French $\&$ Wakker}
\usepackage{graphicx}
\usepackage{subfigure}
\usepackage{amsmath}

%\usepackage{dblfloatfix}

%\usepackage{longtable}
%\usepackage{deluxetable}


\newcommand{\kms}{$\rm km\, s^{-1}$}
\newcommand{\HI}{\mbox{H\,{\sc i}} }

%\newcommand{\HI}{H\,{\sc i}}


\newcommand{\I}{\,{\sc i}}
\newcommand{\II}{\,{\sc ii}}
\newcommand{\III}{\,{\sc iii}}
\newcommand{\IV}{\,{\sc iv}}
\newcommand{\V}{\,{\sc v}}
\newcommand{\VI}{\,{\sc vi}}


\graphicspath{{figures//}}

\begin{document}

\title{The environmental dependence of low-$z$ Ly$\alpha$ absorption}

%Do Ly$\alpha$ absorbers co-rotate with galaxies?}

\author{David M. French, Bart P. Wakker}

\affil{Department of Astronomy, University of Wisconsin, Madison, WI 53706, USA}

\begin{abstract}
We present the results of a large-scale study of the Ly$\alpha$-probed CGM of nearby galaxies. We have identified \textbf{XXXX} Ly$\alpha$ absorbers in the redshift range $0 \leq z \leq 0.033$ and correlated their positions with the surrounding galaxy environment, leading to a sample of \textbf{XXXX} Ly$\alpha$ component-galaxy pairs, representing the largest-to-date dataset of it's kind. By employing the likelihood-based matching scheme of \cite{french2017}, we quantify the absorber-galaxy spacial correlation and identify 4 distinct absorber sub-samples. We find that absorber equivalent width and Doppler-b parameter are enhanced with increasing proximity to galaxies.

\end{abstract}


\keywords{galaxies:intergalactic medium, galaxies:evolution, galaxies:halos, quasars: absorption lines}


\section{INTRODUCTION}



\section{DATA ANALYSIS}


\section{RESULTS}

In our sample of XXXXXX QSOs we have detected XXXXXX Ly$\alpha$ absorbers. Figure \textbf{FULL SKY MAP} shows an all-sky map of the positions of all absorbers split into 4 velocity bins ($v_{\rm Ly\alpha} = 0 - 2500$, $2501 - 5000$, $5001 - 7500$, and $7501 - 10,000$ \kms). Comparing this to an all-sky map of the galaxies within this range shown in Figure \textbf{ALL SKY GALAXIES} (see \textbf{Chapter 1}), we see that the Ly$\alpha$ absorbers broadly trace the locations of the galaxies. If the current $\Lambda$ Cold-Dark Matter ($\Lambda$CDM) cosmology is to be believed, this should not be remarkably surprising. Galaxies and the gas traced by $\rm Ly\alpha$ absorbers should both follow the underlying potential produced by the Dark Matter, and should therefore be found in similar places. Beyond this big-picture result however, we want to know how the absorbers react to the presence of the galaxies on a more local scale.



As first introduced in \cite{french2017}, we employ a unique likelihood-method for objectively matching absorbers with nearby galaxies. We define the likelihood as follows: 

\begin{equation}
\mathcal{L} = A \times e^{-(\frac{\rho}{R_{eff}})^2} \times e^{-(\frac{\Delta v}{v_{norm}})^2},
\end{equation}

where $A$ is a normalization constant, $\rho$ is the impact parameter, $R_{eff}$ is one of two possible ``effective - radii" we use for the galaxy (virial radius and $D^{1.5}$, or diameter to the 1.5 power), $\Delta v$ is the velocity separation between absorber and galaxy heliocentric, and $v_{norm}$ is a velocity normalization (equal to one of 150, 200, or 250). We calculate $\mathcal{L}$ for every absorber-galaxy combination, which then gives us a single number as a three-dimensional proxy for the physical separation between the two. We furthermore explore the results of adjusting the several possible $\mathcal{L}$ normalizations. We calculate $\mathcal{L}$ with $R_{eff}$ equal to $R_{vir}$ and $D^{1.5}$ and $v_{norm}$ equal to 150, 200, and 250. For each of these combinations, we also calculate a variant with $A =1$ and another with $A = 2$ if $R_{eff} \ge \rho$, and $A=1$ otherwise. Table \textbf{TABLE} summarizes the resulting subsets for each of these combinations.




\begin{figure*}[ht!]
        \centering
        \vspace{0pt}
        \includegraphics[width=0.95\textwidth]{hist(EW)_all6_bins10_6alt_min_maxEW_0_10000_err.pdf}
        \caption{\small{The equivalent width (EW) cumulative distribution function for each subset of our Ly$\alpha$ absorber sample. From the top-left corner to the bottom-right the curves are the fully isolated absorbers (grey), the absorbers isolated enough from any galaxy to not be likelihood-matched (brown), the full distribution (black), the absorbers likelihood-matched to a single, non-isolated galaxy (orange), the absorbers matched to a single, isolated galaxy (green), and the absorbers likelihood-matched with two or more galaxies (purple). The shaded region around each curve gives the EW measurement errors.}}
        \vspace{-5pt}
        \label{cdf_ew}
\end{figure*}


\begin{figure*}[ht!]
        \centering
        \vspace{0pt}
        \includegraphics[width=0.95\textwidth]{hist(b)_all6_bins1_6_min_maxEW_0_10000.pdf}
        \caption{\small{The Doppler b-parameter ($b$) cumulative distribution function for each subset of our Ly$\alpha$ absorber sample. From the top-left corner to the bottom-right the curves are the fully isolated absorbers (grey), the absorbers isolated enough from any galaxy to not be likelihood-matched (brown), the full distribution (black), the absorbers likelihood-matched to a single, non-isolated galaxy (orange), the absorbers matched to a single, isolated galaxy (green), and the absorbers likelihood-matched with two or more galaxies (purple). The shaded region around each curve gives the EW measurement errors.}}
        \vspace{-5pt}
        \label{cdf_b}
\end{figure*}



\section{DISCUSSION}


\section{SUMMARY}


 \begin{deluxetable*}{l l l l l l l l l l l}
%\setlength{\tabcolsep}{0.05in}
\tablecolumns{11}
\tabletypesize{\scriptsize}
%\tablewidth{0pt}
\tablecaption{SALT Galaxy Observations\label{tab:params}}
\tablehead{
\colhead{Galaxy}	&  \colhead{R.A.}	&  \colhead{Dec.}  	&  \colhead{Measured $v_{\rm sys}$}& \colhead{Published $v_{\rm sys}$} & \colhead{Type} &  \colhead{Grating}	&  \colhead{$v_{\rm rot}$}	& \colhead{$v_{\rm rot} / \sin(\emph{i})$}	& \colhead{Obs. Date} & \colhead{$T_{\rm exp}$}  \\
			  	&          			&  			 	& \colhead{(\kms)}  				& \colhead{(\kms)}  		     	   &				& 				&  \colhead{(\kms)}  		& \colhead{(\kms)}					&					& \colhead{(ks)} }
\colnumbers
\startdata
 CGCG039-137 	& 11 21 27.0		& +03 26 41.7		& $6918 \pm24$\				&	$6902 \pm 52^{a}$	& Scd		& PG2300			& $132 \pm 16$	& $139 \pm 26$			& 05 11 2016		& 700	\\ %done
 \hline
\enddata
\tablecomments{SALT targeted galaxies. Columns are as follows: 1) the galaxy name, 2), 3) R.A., Dec. in J2000, 4) galaxy systemic velocity, 5) morphological type (RC3), 6) RSS grating used, 7) approaching side velocity, 8) receding side velocity, 9) observation date, and 10) exposure time}
%\tablenotetext{a}{\cite{sdssDR3}} 
%\tablenotetext{b}{\cite{6dFDR3}}
%\tablenotetext{c}{\cite{RC3}}
%\tablenotetext{d}{\cite{mathewson1996}}
%\tablenotetext{e}{\cite{koribalski2004}}
%\tablenotetext{f}{\cite{RC3}}
%\tablenotetext{g}{\cite{lu1993}}
%\tablenotetext{h}{\cite{grogin1998}}
%\tablenotetext{i}{\cite{koribalski2004}}
%\tablenotetext{j}{\cite{RC3}}
%\tablenotetext{k}{\cite{dinella1996}}
%\tablenotetext{l}{\cite{giovanelli1997}}
%\tablerefs{\cite{giovanelli1997}}
\tablerefs{a. \cite{sdssDR3}; b. \cite{6dFDR3}; c. \cite{RC3}; d. \cite{mathewson1996}; e. \cite{koribalski2004}; f. \cite{lu1993}; g. \cite{grogin1998}; h. \cite{dinella1996}; i, \cite{giovanelli1997}}
 \label{salt_targets}
\end{deluxetable*}



%%\begin{eqnarray}
%%	\nonumber
%%	\sigma^2 = \left( \frac{\partial v_{rot}}{\partial \lambda_{obs}} \right)^2 (\Delta \lambda_{obs})^2 + \\
%%	\nonumber
%%	\left(\frac{\partial v_{rot}}{\partial v_{sys}} \right)^2 (\Delta v_{sys})^2 + \\
%%	\left( \frac{\partial v_{rot}}{\partial i} \right)^2 (\Delta i)^2,
%%\end{eqnarray}



\startlongtable
\begin{deluxetable*}{l l l l l l l l l l l}
%\setlength{\tabcolsep}{0.1in}
\tablecolumns{11}
\tabletypesize{\scriptsize}
%\tablewidth{1pt}
\tablecaption{Halo Model Results and Ly$\alpha$ Absorption Properties\label{models}}
\tablehead{
\colhead{$\#$}	&\colhead{Galaxy}  	&  \colhead{Target} 	&  \colhead{$\rho$ }  &  \colhead{Az.}      & \colhead{$v_{\rm sys}$}&  \colhead{$v_{\rm rot}$ }   &  \colhead{$v_{\rm Ly\alpha}$} & \colhead{$W_{\rm Ly\alpha}$} & \colhead{Cyl. Model}  & \colhead{NFW Model }  \\
			  			&				&          			&  \colhead{(kpc)}  	& \colhead{(Deg.)}  & \colhead{(\kms)}	     & \colhead{(\kms)}  		& \colhead{(\kms)}  		      	&  \colhead{(m\AA)}  			  & \colhead{(\kms)} 	       & \colhead{(\kms)} }
\colnumbers
\startdata
1  &        CGCG039-137  &   RX\_J1121.2+0326  &         99  &     71  &  6918  &      139  &       6975  &           678  &            6882 - 7055  & 6881 - 7082    \\
\enddata
\tablecomments{Comments.}
\end{deluxetable*}

%\begin{figure*}[ht!]
%        \centering
%        \vspace{0pt}
%        \includegraphics[width=0.95\textwidth]{SALTmap_velstrict_False_non_True_Lstar_0-100_minsep_False_inclim_00.pdf}
%        \caption{\small{A map of the locations of each absorber normalized with respect to the galaxy virial radius. The color and style of each point indicates the line-of-sight velocity compared to that of the rotation of the nearby galaxy. Blue diamonds indicate co-rotation, red crosses indicate anti-rotation, and grey circles indicate cases where either is possible due to a combination of orientation and velocity uncertainties. The size of each point is scaled to reflect the EW of the absorber. Concentric rings indicate distances of 1, 2, and 3 $R_{\rm vir}$. All galaxies are rotated to PA = 90 or 270, such that their major axis' are horizontal and their approaching side is on the left as indicated. The number identifiers correspond to the system number given in column (1) of Table \ref{model}.}}
%        \vspace{-5pt}
%        \label{full_map}
%\end{figure*}


%%\begin{figure*}
%%\centering
%%  \subfigure[]{\includegraphics[width=0.5\linewidth]{SALTmap_velstrict_False_non_True_Lstar_0-100_minsep_False_zoom_10_inclim_00.pdf}}{\label{zoom_map}}
%%  \subfigure[]{\includegraphics[width=0.48\linewidth]{SALTmap_NFW_model_velstrict_True_non_True_Lstar_0-100_minsep_False_inclim_00.pdf}\label{nfw_map}}
%%  \caption{\small{Maps of the locations of each absorber normalized with respect to the galaxy virial radius. \textbf{Left:} A zoom in showing only those systems within $1 R_{\rm vir}$. The color and style of each point indicates the line-of-sight velocity compared to that of the rotation of the nearby galaxy. \textbf{Right:} The color and style of each point indicates the NFW rotation model results for each absorber with a $v_{\rm Ly\alpha} \leq v_{\rm rot}$ constraint imposed. Concentric rings indicate distances of 1, 2, and 3 $R_{vir}$. \textbf{Both:} Blue diamonds indicate co-rotation, red crosses indicate anti-rotation, and grey circles indicate cases where either is possible due to a combination of orientation and velocity uncertainties. The size of each point is scaled to reflect the EW of the absorber. All galaxies are rotated to PA = 90 or 270, such that their major axis' are horizontal and their approaching side is on the left as indicated. The number identifiers correspond to the system number given in column (1) of Table \ref{model}.}}
%%\vspace{0pt}
%%\end{figure*}




\acknowledgements

D. M. F. thanks \textbf{A BUNCH OF PEOPLE}.This research has made use of the NASA/IPAC Extragalactic Database (NED) which is operated by the Jet Propulsion Laboratory, California Institute of Technology, under contract with the National Aeronautics and Space Administration. Based on observations with the NASA/ESA \textit{Hubble Space Telescope}, obtained at the Space Telescope Science Institute (STScI), which is operated by the Association of Universities for Research in Astronomy, Inc., under NASA contract NAS 5-26555. \textbf{SALT ACKNOWLEDGEMENT}. Spectra were retrieved from the Barbara A. Mikulski Archive for Space Telescopes (MAST) at STScI. Over the course of this study, D.M.F. and B.P.W. were supported by grant XXXX

%AST-1108913, awarded by the US National Science Foundation, and by NASA grants \textit{HST}-AR-12842.01-A, \textit{HST}-AR-13893.01-A, and \textit{HST}-GO-14240 (STScI). 

\facility{HST (COS), SALT (RSS)}
\clearpage

%\nocite{*}
%\bibliography{rotation_bib}
%\bibliography{/Users/frenchd/Research/bib}{}
\bibliography{/Users/frenchd/Research/inclination/git_inclination/bib}{}
\bibliographystyle{apj}

\clearpage

\appendix

\end{document}
