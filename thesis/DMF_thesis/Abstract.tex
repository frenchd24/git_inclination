\chapter*{Abstract}
\addcontentsline{toc}{section}{Abstract}

Current Lambda Cold-Dark-Matter cosmology predicts the growth of large scale structure from small, initial perturbations in the dark matter potential distribution. Gas in the Universe follows this dark matter potential, collapsing along a series of filaments and knots, and eventually forming galaxies. The present-day Universe is left with a rich kaleidoscope of galaxies of different types, colors, and sizes. Observations suggest, however, that the gas from which these galaxies are built continues to play a pivotal role in their evolution. Numerous studies have already found strong evidence for the spatial correlation of neutral hydrogen (\HI; $\rm Ly\alpha$ absorption being our tracer of choice here) in the intergalactic medium (IGM) and galaxies. This thesis examines how intergalactic \HI~interacts with, is influenced by, and depends on the nearby galaxies lying within it. 

%Understanding the complex relationship between these two sources of baryonic matter remains key to understanding their evolution over cosmic time. 

This study is made possible by correlating the positions of archival Cosmic Origins Spectrograph (COS; Hubble Space Telescope) sightlines toward background quasi-stellar objects (QSOs) with the distributions of known galaxies within the nearby, $cz \leq 10,000$ \kms~Universe. To enable this, I compile a catalog of all known galaxies in this redshift range, and homogenized measurements of their diameters, axis ratios, position angles, and magnitudes. This effort ensures that I can compare galaxy measurements from different sources with confidence. 

I introduce a novel likelihood method to automate the process of matching galaxies from this catalog with nearby absorption and also to quantify the relative isolation of these absorbers versus their proximity to galaxies. To test this method, I built a pilot sample with 33 QSO sightines chosen for their proximity to large ($D \ge 25$ kpc) galaxies. In each I identify all $\rm Ly\alpha$ lines within $cz \leq 10,000$ \kms~and match each line with the highest-likelihood galaxy. I discover a preference for $\rm Ly\alpha$ to be detected near high inclination galaxies at a $3.6\sigma$ significance level. I attribute this to the combination of a $< 100\%$ covering fraction and flattened, non-spherical $\rm Ly\alpha$ galaxy halos, which would result in a higher probability of detection due to the increased pathlength through an edge-on, disky halo.

To test for a rotational $\rm Ly\alpha$ component, I obtain long-slit rotation curve observations for 12 galaxies with the Southern African Large Telescope. I combine this sample with an additional 17 galaxies from the literature, and develop a halo rotation model based on observed rotation curves to de-project the halo gas rotation probed by QSO absorption-line spectroscopy. I find that a rotating halo with velocities declining with distance based on NFW-profile fits results in the highest $\rm Ly\alpha$ absorber co-rotation fraction, and that this co-rotation fraction declines as a function of galaxy luminosity. This result matches the predictions of numerous simulations that cold-mode accretion both dominates in lower-mass galaxies, and carries coherent angular momentum deep into galaxy halos.

Finally, I produce a sample of 1135 $\rm Ly\alpha$ absorbers from 264 individual QSO spectra taken by COS and employ our likelihood method to sort them by galaxy proximity. I find that $\rm Ly\alpha$ equivalent width depends strongly on galaxy proximity, with $\gtrsim 95\%$ of EW $\rm \leq 400~ m\AA$ appearing farther than 500 kpc and 400 \kms~ from any galaxy, compared to $\lesssim75\%$ of those near to at least one galaxy. I confirm our pilot study findings of a higher detection fraction near highly inclined galaxies ($\sim4.6 \sigma$), and report the first detection of an azimuth angle dependence for $\rm Ly\alpha$ detection. As has been previously reported for multiple metal lines, $\rm Ly\alpha$ absorbers are more commonly detected near the major and minor axes of galaxies ($\sim 3.3\sigma$ significance). 



